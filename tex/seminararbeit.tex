% !TeX root = ./seminararbeit.tex
%% Preambel
\documentclass[conference,compsoc,final,a4paper]{IEEEtran}
\usepackage[utf8]{inputenx}

%% Bitte legen Sie hier den Titel und den Autor der Arbeit fest
\newcommand{\autoren}[0]{Mundhenke, Moritz}
\newcommand{\dokumententitel}[0]{Spectre and Cloud : An evaluation of threats in shared computation environments }

% Hie muss normalerweise nichts angepasst werden
\usepackage[pdftex]{graphicx}
\graphicspath{{img/}}
\DeclareGraphicsExtensions{.pdf,.jpeg,.jpg,.png}
\usepackage[cmex10]{amsmath}
\usepackage{algorithmic}
\usepackage{array}
\usepackage{dblfloatfix}
\usepackage{url}
\usepackage[autostyle=true]{csquotes}
\usepackage[backend=biber]{biblatex}
\usepackage{booktabs}
\usepackage{xcolor}
\usepackage{listings}             % Source Code listings
\usepackage[printonlyused]{acronym}
\usepackage{fancyvrb}

% Farben definieren
\definecolor{linkblue}{RGB}{0, 0, 100}
\definecolor{linkblack}{RGB}{0, 0, 0}
\definecolor{darkgreen}{RGB}{14, 144, 102}
\definecolor{darkblue}{RGB}{0,0,168}
\definecolor{darkred}{RGB}{128,0,0}
\definecolor{comment}{RGB}{63, 127, 95}
\definecolor{javadoccomment}{RGB}{63, 95, 191}
\definecolor{keyword}{RGB}{108, 0, 67}
\definecolor{type}{RGB}{0, 0, 0}
\definecolor{method}{RGB}{0, 0, 0}
\definecolor{variable}{RGB}{0, 0, 0}
\definecolor{literal}{RGB}{31,0, 255}
\definecolor{operator}{RGB}{0, 0, 0}

\usepackage[english]{betababel}
\usepackage[
      unicode=true,
      hypertexnames=false,
      colorlinks=true,
      colorlinks=false,
      linkcolor=darkblue,
      citecolor=darkblue,
      urlcolor=darkblue,
      pdftex
   ]{hyperref}
%	 \PrerenderUnicode{ü}

% Einstellungen für Quelltexte
\lstset{
    xleftmargin=0.1cm,
    basicstyle=\scriptsize\ttfamily,
    keywordstyle=\color{keyword},
    identifierstyle=\color{variable},
    commentstyle=\color{comment},
    stringstyle=\color{literal},
    tabsize=2,
    lineskip={2pt},
    columns=flexible,
    inputencoding=utf8,
    captionpos=b,
    breakautoindent=true,
    breakindent=2em,
    breaklines=true,
    prebreak=,
    postbreak=,
    numbers=none,
    numberstyle=\tiny,
    showspaces=false,      % Keine Leerzeichensymbole
    showtabs=false,        % Keine Tabsymbole
    showstringspaces=false,% Leerzeichen in Strings
    morecomment=[s][\color{javadoccomment}]{/**}{*/},
    literate={Ö}{{\"O}}1 {Ä}{{\"A}}1 {Ü}{{\"U}}1 {ß}{{\ss}}2 {ü}{{\"u}}1 {ä}{{\"a}}1 {ö}{{\"o}}1
}

\hypersetup{
    pdftitle={\dokumententitel},
    pdfauthor={\autoren},
    pdfdisplaydoctitle=true,
    hidelinks
}

% Makros für typographisch korrekte Abkürzungen
\newcommand{\zb}[0]{z.\,B.\ }
\newcommand{\dahe}[0]{d.\,h.\ }
\newcommand{\ua}[0]{u.\,a.\ }

% Wo liegt Sourcecode?
\newcommand{\srcloc}{src/}

% Literatur einbinden
\addbibresource{literatur.bib}
 % Weitere Einstellungen aus einer anderen Datei lesen

\begin{document}

% Titel des Dokuments
\title{\dokumententitel}

% Namen der Autoren
\author{
  \IEEEauthorblockN{\autoren}
  \IEEEauthorblockA{
    Hochschule Mannheim\\
    Fakultät für Informatik\\
    Paul-Wittsack-Str. 10,
    68163 Mannheim
    }
}

% Titel erzeugen
\maketitle
\thispagestyle{plain}
\pagestyle{plain}

% Eigentliches Dokument beginnt hier
% ----------------------------------------------------------------------------------------------------------

% Kurze Zusammenfassung des Dokuments
\begin{abstract}
\end{abstract}

% Inhaltsverzeichnis erzeugen
\tableofcontents

% Abschnitte mit \section, Unterabschnitte mit \subsection und
% Unterunterabschnitte mit \subsubsection
% -------------------------------------------------------
\section{Einleitung}

% -------------------------------------------------------
\section{Modern computation hardware and infrastructure}
\subsection{Memory management}
* Caches
\subsection{Processor pipelines and speculative execution}
Since the turn of the millennium, clock speeds of high performance CPU's have stagnated because the increased energy consumption and resulting heat generation got
too high. \cite{fog2012microarchitecture} Multi-core architectures became common however even today many, if not most,
workloads rely heavily on single thread performance. Therefore to increase the speed of single thread execution, without increasing the clock speed,
more instructions have to be executed in a single cycle. To achieve this goal many techniques like instruction splitting, fusion,
and simultaneous instruction execution are used \cite{fog2012microarchitecture}. These optimizations require more complex processing of the instruction stream which
is handled by the processors instruction pipeline. This has resulted in higher pipeline memory requirements as well as a longer pipelines in terms of clock cycles
required for an instruction to complete execution. However since programs use conditional branches and loops these long pipeline can become invalid because the CPU
cannot know the destination of a conditional jump without first evaluating the condition. This results in a pipeline flush that wastes valuable execution time while
the pipeline is refilled. To mitigate the impact of this issue the processor can store previous outcomes of branches and use this to predict future executions of the
branch. With branch prediction only a miss-prediction will result in a pipeline flush. The CPU runs the instruction of the predicted branch and either commits
the results if the prediction was correct or discards them if was not.
This process is called speculative execution. \cite{kocher2018spectre}
\subsection{Virtual and shared memory}

\subsection{Docker}
\subsubsection*{Containerization concept}
Docker allows isolation of processes in containers. These containers hide other processes running on the host and provide the process with its own virtual filesystem
and network. This means containers only share the hosts kernel which stands in contrast with classical VMs which have to run a separate OS for each virtual
machine. To create a container an image is used. The image contains all required files and serves as a \enquote{blueprint} for the container. Due to the lightweight
nature of containers it possible and best practice to isolate every application in its own container. For example a classic LAMP stack would be use a MySQL
and an Apache container. This has multiple benefits. Developers can build and test container images on their own machine which will behave
identical when run on a production server. Docker also allows easy horizontal scaling of applications. The LAMP stack created before could be extended by
creating an additional Apache container and an nginx container to load balance between the two Apache servers.
\subsubsection*{File-system layers}
To reduce redundancy docker images are divided into layers. For example an Apache PHP application would consist of 3
major layers. The first contains a minimal debian image providing basic libraries and tools. The second layer contains Apache and its dependencies. The final
layer would then have a copy of the applications php source files. When a container is started from this image, the layers are stacked on top of each other
using a union filesystem like OverlayFS\cite{overlayfs}. Whenever the application attempts to read a file the kernel checks each layer from top to bottom
returning the first result. This allows overriding of files on lower layers. Docker also adds a final non-persistent layer to the container which the application
running inside uses to write files.
\subsection{Cloud}

% -------------------------------------------------------
\section{Spectre}
\subsection{Side-channels}
Computer programs can, on an abstract level, be described with the input-process-output(IPO) model. An input will produce an output defined by the algorithms
of the program. This output is deterministic, meaning input a will always result in output b. However in real life systems processes can produce additional outputs
which are not intended by its programmer. For example a program which calculates the factorial of a number takes longer proportional to the input number.
An attacker could therefore infer the input number from the execution time without knowing the result. More complex timing-attacks have been used to
extract private keys or other secrets of various cryptographic algorithms. \cite{bernstein2005cache}\cite{kocher1996timing}
Similarly even a systems power consumption can be used as a side-channel to retrieve secret data. \cite{kocher1999differential}
While these two side-channels can add unintended output to a process they are within the programmers control. Execution time can be increased arbitrarily through
idling or sleeping and the system's power consumption can be increased by running pointless code. Other side-channels however are completely outside of the
control of the programmer. For example a program has very limited or no direct control over the systems cache state or which ALU will compute a calculation.
\subsection{Exploiting speculative execution}
To execute a spectre exploit the attacker first has to find a vulnerable instruction sequence in the victim program. The sequence will leak secret data into the chosen
side-channel during speculative execution. Therefore spectre-style attacks can be differentiated from each other by the how speculative execution is achieved
and what side-channel is used. \cite{kocher2018spectre} The example of Variant 1 of the original spectre attack uses conditional branches to achieve speculative
execution and leaks data through the processors cache state. The conditional branch is an array bounds check \lstinline|if (x < array1_size)|. The branch prediction
can be miss-trained by repeatedly calling the victim function with an in-bounds \lstinline|x| causing it to predict the branch as true in the future. Then the
attacker flushes \lstinline|array1_size| from the processors cache and calls the victim function again. This causes the branch to be execute speculatively
until \lstinline|array1_size| is retrieved from memory. To leak the victims secret the attacker chooses an \lstinline|x| that is out of bounds. The victim
now reads from \lstinline|array1| using the malicious, out-of-bounds address. To successfully execute the attack sequence the victim also has to use the result to
access a second array. This causes the processor to load a memory address into the cache which is based of the previous out-of-bounds access of \lstinline|array[x]|.
The attacker can now time their own access of the possible memory addresses revealing which address can be read fastest and was therefore retrieved by the victims
speculative execution, revealing the (secret) value of \lstinline|array[x]|.
% -------------------------------------------------------
\section{Spectre in shared cloud infrastructure}

\subsection{Shared hypervisor}

\subsection{Shared Docker host}

\subsection{FaaS -- Function as a Service}

% -------------------------------------------------------
\section{Prevention and detection}
\subsection{In-memory encryption}
\subsection{Heuristic detection}

% --------------------------------------------------------------------
\section*{Abkürzungen}
\addcontentsline{toc}{section}{Abkürzungen}

% Die längste Abkürzung wird in die eckigen Klammern
% bei \begin{acronym} geschrieben, um einen hässlichen
% Umbruch zu verhindern
% Sie müssen die Abkürzungen selbst alphabetisch sortieren!
\begin{acronym}[IEEE]
\end{acronym}

% Literaturverzeichnis
\addcontentsline{toc}{section}{Literatur}
\printbibliography

\end{document}
